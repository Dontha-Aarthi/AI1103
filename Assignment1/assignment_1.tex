\documentclass[journal,12pt,twocolumn]{IEEEtran}

\usepackage{setspace}
\usepackage{gensymb}
\singlespacing
\usepackage[cmex10]{amsmath}

\usepackage{amsthm}

\usepackage{mathrsfs}
\usepackage{txfonts}
\usepackage{stfloats}
\usepackage{bm}
\usepackage{cite}
\usepackage{cases}
\usepackage{subfig}

\usepackage{longtable}
\usepackage{multirow}

\usepackage{enumitem}
\usepackage{mathtools}
\usepackage{steinmetz}
\usepackage{tikz}
\usepackage{circuitikz}
\usepackage{verbatim}
\usepackage{tfrupee}
\usepackage[breaklinks=true]{hyperref}
\usepackage{graphicx}
\usepackage{tkz-euclide}

\usetikzlibrary{calc,math}
\usepackage{listings}
    \usepackage{color}                                            %%
    \usepackage{array}                                            %%
    \usepackage{longtable}                                        %%
    \usepackage{calc}                                             %%
    \usepackage{multirow}                                         %%
    \usepackage{hhline}                                           %%
    \usepackage{ifthen}                                           %%
    \usepackage{lscape}     
\usepackage{multicol}
\usepackage{chngcntr}

\DeclareMathOperator*{\Res}{Res}

\renewcommand\thesection{\arabic{section}}
\renewcommand\thesubsection{\thesection.\arabic{subsection}}
\renewcommand\thesubsubsection{\thesubsection.\arabic{subsubsection}}

\renewcommand\thesectiondis{\arabic{section}}
\renewcommand\thesubsectiondis{\thesectiondis.\arabic{subsection}}
\renewcommand\thesubsubsectiondis{\thesubsectiondis.\arabic{subsubsection}}


\hyphenation{op-tical net-works semi-conduc-tor}
\def\inputGnumericTable{}                                 %%

\lstset{
%language=C,
frame=single, 
breaklines=true,
columns=fullflexible
}
\begin{document}


\newtheorem{theorem}{Theorem}[section]
\newtheorem{problem}{Problem}
\newtheorem{proposition}{Proposition}[section]
\newtheorem{lemma}{Lemma}[section]
\newtheorem{corollary}[theorem]{Corollary}
\newtheorem{example}{Example}[section]
\newtheorem{definition}[problem]{Definition}

\newcommand{\BEQA}{\begin{eqnarray}}
\newcommand{\EEQA}{\end{eqnarray}}
\newcommand{\define}{\stackrel{\triangle}{=}}
\bibliographystyle{IEEEtran}
\raggedbottom
\setlength{\parindent}{0pt}
\providecommand{\mbf}{\mathbf}
\providecommand{\pr}[1]{\ensuremath{\Pr\left(#1\right)}}
\providecommand{\qfunc}[1]{\ensuremath{Q\left(#1\right)}}
\providecommand{\sbrak}[1]{\ensuremath{{}\left[#1\right]}}
\providecommand{\lsbrak}[1]{\ensuremath{{}\left[#1\right.}}
\providecommand{\rsbrak}[1]{\ensuremath{{}\left.#1\right]}}
\providecommand{\brak}[1]{\ensuremath{\left(#1\right)}}
\providecommand{\lbrak}[1]{\ensuremath{\left(#1\right.}}
\providecommand{\rbrak}[1]{\ensuremath{\left.#1\right)}}
\providecommand{\cbrak}[1]{\ensuremath{\left\{#1\right\}}}
\providecommand{\lcbrak}[1]{\ensuremath{\left\{#1\right.}}
\providecommand{\rcbrak}[1]{\ensuremath{\left.#1\right\}}}
\theoremstyle{remark}
\newtheorem{rem}{Remark}
\newcommand{\sgn}{\mathop{\mathrm{sgn}}}
\providecommand{\abs}[1]{\left\vert#1\right\vert}
\providecommand{\res}[1]{\Res\displaylimits_{#1}} 
\providecommand{\norm}[1]{\left\lVert#1\right\rVert}
%\providecommand{\norm}[1]{\lVert#1\rVert}
\providecommand{\mtx}[1]{\mathbf{#1}}
\providecommand{\mean}[1]{E\left[ #1 \right]}
\providecommand{\fourier}{\overset{\mathcal{F}}{ \rightleftharpoons}}
%\providecommand{\hilbert}{\overset{\mathcal{H}}{ \rightleftharpoons}}
\providecommand{\system}{\overset{\mathcal{H}}{ \longleftrightarrow}}
	%\newcommand{\solution}[2]{\textbf{Solution:}{#1}}
\newcommand{\solution}{\noindent \textbf{Solution: }}
\newcommand{\cosec}{\,\text{cosec}\,}
\providecommand{\dec}[2]{\ensuremath{\overset{#1}{\underset{#2}{\gtrless}}}}
\newcommand{\myvec}[1]{\ensuremath{\begin{pmatrix}#1\end{pmatrix}}}
\newcommand{\mydet}[1]{\ensuremath{\begin{vmatrix}#1\end{vmatrix}}}
\numberwithin{equation}{subsection}
\makeatletter
\@addtoreset{figure}{problem}
\makeatother
\let\StandardTheFigure\thefigure
\let\vec\mathbf
\renewcommand{\thefigure}{\theproblem}
\def\putbox#1#2#3{\makebox[0in][l]{\makebox[#1][l]{}\raisebox{\baselineskip}[0in][0in]{\raisebox{#2}[0in][0in]{#3}}}}
     \def\rightbox#1{\makebox[0in][r]{#1}}
     \def\centbox#1{\makebox[0in]{#1}}
     \def\topbox#1{\raisebox{-\baselineskip}[0in][0in]{#1}}
     \def\midbox#1{\raisebox{-0.5\baselineskip}[0in][0in]{#1}}
\vspace{3cm}
\title{Assignment 1}
\author{Dontha Aarthi - CS20BTECH11015}
\maketitle
\newpage
\bigskip
\renewcommand{\thefigure}{\theenumi}
\renewcommand{\thetable}{\theenumi}
Download all python codes from 
\begin{lstlisting}

\end{lstlisting}
%
and latex-tikz codes from 
%
\begin{lstlisting}

\end{lstlisting}
\section{Problem}
Question 2.1:\\
Bag I contains 3 red and 4 black balls and
Bag II contains 4 red and 5 black balls. One
ball is transferred from Bag I to Bag II and
then a ball is drawn from Bag II. The ball
so drawn is found to be red in colour. Find
the probability that the transferred ball is black.


\section{Solution}

Let X $\in$ $ \{1,2\} $ represent the bags and Y $\in$ $\{0,1\}$ where 1 denotes black and 0 denotes red.\\
\begin{align}
    Pr(Y=1/X=1)=\frac{4}{7}\\
    Pr(Y=0/X=1)=\frac{3}{7}
\end{align}
Let $E_1$ denote the event of black ball being transferred from bag 1,\\ $E_2$ denote the event of red being transferred from bag 1 and\\ T be the event of red ball being drawn from bag 2 after transferring a ball from bag 1.\\
We can find the probability of $E_1$ and $E_2$ using 2.0.1 and 2.0.2
\begin{align}
    Pr(E_1)=Pr(Y=1/X=1)=\frac{4}{7}\\
    Pr(E_2)=Pr(Y=0/X=1)=\frac{3}{7}
\end{align}
Probability that red ball is drawn from bag 2 and if red ball is transferred from bag 1 is
\begin{align}
    Pr(T/E_2)=\frac{5}{10}=\frac{1}{2}
\end{align}
Since red ball is being transferred, the number of red balls in bag 2 becomes 5 and total number of balls also increases by 1, i.e, 10.\\
And Probability that red ball is drawn from bag 2 and if black ball is transferred from bag 1 is
\begin{align}
    Pr(T/E_1)=\frac{4}{10}=\frac{2}{5}
\end{align}
Since black ball is being transferred, the number of red balls in bag 2 remains same and only total number of balls increases by 1, i.e, 10.\\

Now, we have to find probability of black ball being transferred from bag 1 to bag 2 if a red ball is being drawn from bag 2.\\
Using Baye's theorem,
\begin{align}
    Pr(E_1/T)=\frac{Pr(E_1T)}{Pr(T)}
\end{align}
\begin{align}
    Pr(E_1/T)=\frac{Pr(E_1)Pr(T/E_1)}{Pr(E_1)Pr(T/E_1)+Pr(E_2)Pr(T/E_2)}
\end{align}
On substituting the values of Pr($E_1$), Pr($E_2$), Pr(T/$E_1$) and Pr(T/$E_2$) in equation 2.0.8, we get

    \[Pr(E_1/T)=\frac{\frac{4}{7}.\frac{2}{5}}{\frac{4}{7}.\frac{2}{5}+\frac{3}{7}.\frac{1}{2}}\]
    \[\implies Pr(E_1/T)=\frac{\frac{8}{35}}{\frac{8}{35}+\frac{3}{14}}\]
    \begin{align}
        \implies Pr(E_1/T)=\frac{16}{31}
    \end{align}
    




\end{document}
