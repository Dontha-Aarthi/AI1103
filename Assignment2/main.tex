\documentclass[journal,12pt,twocolumn]{IEEEtran}

\usepackage{setspace}
\usepackage{gensymb}
\singlespacing
\usepackage[cmex10]{amsmath}

\usepackage{amsthm}

\usepackage{mathrsfs}
\usepackage{txfonts}
\usepackage{stfloats}
\usepackage{bm}
\usepackage{cite}
\usepackage{cases}
\usepackage{subfig}

\usepackage{longtable}
\usepackage{multirow}

\usepackage{enumitem}
\usepackage{mathtools}
\usepackage{steinmetz}
\usepackage{tikz}
\usepackage{circuitikz}
\usepackage{verbatim}
\usepackage{tfrupee}
\usepackage[breaklinks=true]{hyperref}
\usepackage{graphicx}
\usepackage{tkz-euclide}

\usetikzlibrary{calc,math}
\usepackage{listings}
    \usepackage{color}                                            %%
    \usepackage{array}                                            %%
    \usepackage{longtable}                                        %%
    \usepackage{calc}                                             %%
    \usepackage{multirow}                                         %%
    \usepackage{hhline}                                           %%
    \usepackage{ifthen}                                           %%
    \usepackage{lscape}     
\usepackage{multicol}
\usepackage{chngcntr}

\DeclareMathOperator*{\Res}{Res}

\renewcommand\thesection{\arabic{section}}
\renewcommand\thesubsection{\thesection.\arabic{subsection}}
\renewcommand\thesubsubsection{\thesubsection.\arabic{subsubsection}}

\renewcommand\thesectiondis{\arabic{section}}
\renewcommand\thesubsectiondis{\thesectiondis.\arabic{subsection}}
\renewcommand\thesubsubsectiondis{\thesubsectiondis.\arabic{subsubsection}}


\hyphenation{op-tical net-works semi-conduc-tor}
\def\inputGnumericTable{}                                 %%

\lstset{
%language=C,
frame=single, 
breaklines=true,
columns=fullflexible
}
\begin{document}

\newcommand{\BEQA}{\begin{eqnarray}}
\newcommand{\EEQA}{\end{eqnarray}}
\newcommand{\define}{\stackrel{\triangle}{=}}
\bibliographystyle{IEEEtran}
\raggedbottom
\setlength{\parindent}{0pt}
\providecommand{\mbf}{\mathbf}
\providecommand{\pr}[1]{\ensuremath{\Pr\left(#1\right)}}
\providecommand{\qfunc}[1]{\ensuremath{Q\left(#1\right)}}
\providecommand{\sbrak}[1]{\ensuremath{{}\left[#1\right]}}
\providecommand{\lsbrak}[1]{\ensuremath{{}\left[#1\right.}}
\providecommand{\rsbrak}[1]{\ensuremath{{}\left.#1\right]}}
\providecommand{\brak}[1]{\ensuremath{\left(#1\right)}}
\providecommand{\lbrak}[1]{\ensuremath{\left(#1\right.}}
\providecommand{\rbrak}[1]{\ensuremath{\left.#1\right)}}
\providecommand{\cbrak}[1]{\ensuremath{\left\{#1\right\}}}
\providecommand{\lcbrak}[1]{\ensuremath{\left\{#1\right.}}
\providecommand{\rcbrak}[1]{\ensuremath{\left.#1\right\}}}
\theoremstyle{remark}
\newtheorem{rem}{Remark}
\newcommand{\sgn}{\mathop{\mathrm{sgn}}}
\providecommand{\abs}[1]{\vert#1\vert}
\providecommand{\res}[1]{\Res\displaylimits_{#1}} 
\providecommand{\norm}[1]{\lVert#1\rVert}
%\providecommand{\norm}[1]{\lVert#1\rVert}
\providecommand{\mtx}[1]{\mathbf{#1}}
\providecommand{\mean}[1]{E[ #1 ]}
\providecommand{\fourier}{\overset{\mathcal{F}}{ \rightleftharpoons}}
%\providecommand{\hilbert}{\overset{\mathcal{H}}{ \rightleftharpoons}}
\providecommand{\system}{\overset{\mathcal{H}}{ \longleftrightarrow}}
	%\newcommand{\solution}[2]{\textbf{Solution:}{#1}}
\newcommand{\solution}{\noindent \textbf{Solution: }}
\newcommand{\cosec}{\,\text{cosec}\,}
\providecommand{\dec}[2]{\ensuremath{\overset{#1}{\underset{#2}{\gtrless}}}}
\newcommand{\myvec}[1]{\ensuremath{\begin{pmatrix}#1\end{pmatrix}}}
\newcommand{\mydet}[1]{\ensuremath{\begin{vmatrix}#1\end{vmatrix}}}
\numberwithin{equation}{subsection}
\makeatletter
\@addtoreset{figure}{problem}
\makeatother
\let\StandardTheFigure\thefigure
\let\vec\mathbf
\renewcommand{\thefigure}{\theproblem}
\def\putbox#1#2#3{\makebox[0in][l]{\makebox[#1][l]{}\raisebox{\baselineskip}[0in][0in]{\raisebox{#2}[0in][0in]{#3}}}}
     \def\rightbox#1{\makebox[0in][r]{#1}}
     \def\centbox#1{\makebox[0in]{#1}}
     \def\topbox#1{\raisebox{-\baselineskip}[0in][0in]{#1}}
     \def\midbox#1{\raisebox{-0.5\baselineskip}[0in][0in]{#1}}
\vspace{3cm}
\title{Assignment 1}
\author{Dontha Aarthi-CS20BTECH11015}
\maketitle
\newpage
\bigskip
\renewcommand{\thefigure}{\theenumi}
\renewcommand{\thetable}{\theenumi}
Download all python codes from 
\begin{lstlisting}
    https://github.com/Dontha-Aarthi/AI1103/blob/main/Assignment2/Codes/assignment2.py
\end{lstlisting}
%
and latex-tikz codes from 
%
\begin{lstlisting}
    https://github.com/Dontha-Aarthi/AI1103/blob/main/Assignment2/main.tex
\end{lstlisting}
\begin{center}
  \section{\textbf{Gate Problem 43}} 
\end{center}
    If calls arrive at a telephone exchange such that
the time of arrival of any call is independent
of the time of arrival of earlier or future calls,
the probability distribution function of the total
number of calls in a fixed time interval will be\\
(A) Poisson\\
(B) Gaussian\\
(C) Exponential\\
(D) Gamma\\
\begin{center}
   \section{\textbf{Solution}}
\end{center}
Lets denote the fixed time interval by [0,$T$].
To find the probability of x number of calls during this time interval, lets divide the interval into $n$ equal of parts of length $\Delta t$.
Let us denote the probability of receiving a call during time interval $\Delta t$ by $p$. Suppose the telephone exchange receives an average of $\lambda$ calls in time interval of length $T$.\\
Hence, we have
\begin{align}
    np=\lambda\\
    \implies p=\frac{\lambda}{n}
\end{align}
Since a call has probability p for arriving at a particular instant $(\Delta t\rightarrow 0)$, and the probability of 1-$p$ for not arriving at a particular instant, the probability of receiving k calls in time interval $T$ is
\begin{align}
   \lim_{n \to \infty}\pr{X=k}=\lim_{n \to \infty}\frac{n!}{k!(n-k)!}\left(\frac{\lambda}{n}\right)^k\left(1-\frac{\lambda}{n}\right)^{n-k}
\end{align}
Pulling out constants $\lambda^k$ and $\frac{1}{k!}$ from right-hand side,
\begin{multline}  
    \lim_{n \to \infty}\pr{X=k}=\\
    \left(\frac{\lambda^k}{k!}\right)\lim_{n \to \infty}\frac{n!}{(n-k)!}\left(\frac{1}{n}\right)^k\left(1-\frac{\lambda}{n}\right)^n\left(1-\frac{\lambda}{n}\right)^{-k}
\end{multline}
Now lets take the limit of right-hand side one term at a time. We’ll do this in three steps. The first step is to find the limit of 
\begin{equation}
\begin{split}
     \lim_{n \to \infty}\frac{n!}{(n-k)!n^k}
     &= \lim_{n \to \infty}\frac{n(n-1)(n-2)..(n-k+1)}{n^k}\\
    & = \lim_{n \to\infty}\left(\frac{n}{n}\right)\left(\frac{n-1}{n}\right)....\left(\frac{n-k+1}{n}\right)\\
   &= \lim_{n \to \infty}\left(1-\frac{1}{n}\right)\left(1-\frac{2}{n}\right)...\left(1-\frac{k-1}{n}\right)\\
    &=1\cdot1\cdot1........1\\
    &=1   
    \end{split}
\end{equation}
Now we have to find the limit of 
\begin{align}
    \lim_{n \to \infty}\left(1-\frac{\lambda}{n}\right)^n 
\end{align}
We know that the definition $e$ is given as 
\begin{align}
    e=\lim_{x \to \infty}\left(1+\frac{1}{x}\right)^x
\end{align}
So, lets replace the value of $-\frac{n}{\lambda}$ by x in (2.0.5), we get
\begin{align}
    \lim_{n \to \infty}\left(1-\frac{\lambda}{n}\right)^n =\lim_{x \to \infty}\left(1+\frac{1}{x}\right)^{x(-\lambda)}=e^{-\lambda}
\end{align}
And the third part is to find the limit of 
\begin{align}
    \lim_{n \to \infty}\left(1-\frac{\lambda}{n}\right)^{-k}
\end{align}
As n approaches infinity, this term becomes $1^{-k}$ which is equal to one.
So,
\begin{align}
    \lim_{n \to \infty}\left(1-\frac{\lambda}{n}\right)^{-k}=1
\end{align}\\

Now on substituting (2.0.5), (2.0.8) and (2.0.10) in equation (2.0.4), we get
\begin{multline}  
    \left(\frac{\lambda^k}{k!}\right)\lim_{n \to \infty}\frac{n!}{(n-k)!}\left(\frac{1}{n}\right)^k\left(1-\frac{\lambda}{n}\right)^n\left(1-\frac{\lambda}{n}\right)^{-k}=\\
    \left(\frac{\lambda^k}{k!}\right)(1)\left(e^{-\lambda}\right)(1)
\end{multline}
This just simplifies into
\begin{align}
    \pr{\lambda,k}=\left(\frac{\lambda^k e^{-\lambda}}{k!}\right)
\end{align}
   (2.0.12) is equal to probability density function of Poisson distribution, which gives us probability of $k$ successes per period, with given parameter of $\lambda$.\\
   
   $\therefore $The probability distribution function of the total
number of calls in a fixed time interval will be \textbf{Poisson} distribution.\\
Answer: Option(A)



\end{document}