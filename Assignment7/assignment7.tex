\documentclass[journal,12pt,twocolumn]{IEEEtran}

\usepackage{setspace}
\usepackage{gensymb}
\singlespacing
\usepackage[cmex10]{amsmath}

\usepackage{amsthm}

\usepackage{mathrsfs}
\usepackage{txfonts}
\usepackage{stfloats}
\usepackage{bm}
\usepackage{cite}
\usepackage{cases}
\usepackage{subfig}

\usepackage{longtable}
\usepackage{multirow}

\usepackage{enumitem}
\usepackage{mathtools}
\usepackage{steinmetz}
\usepackage{tikz}
\usepackage{circuitikz}
\usepackage{verbatim}
\usepackage{tfrupee}
\usepackage[breaklinks=true]{hyperref}
\usepackage{graphicx}
\usepackage{tkz-euclide}

\usetikzlibrary{calc,math}
\usepackage{listings}
    \usepackage{color}                                            %%
    \usepackage{array}                                            %%
    \usepackage{longtable}                                        %%
    \usepackage{calc}                                             %%
    \usepackage{multirow}                                         %%
    \usepackage{hhline}                                           %%
    \usepackage{ifthen}                                           %%
    \usepackage{lscape}     
\usepackage{multicol}
\usepackage{chngcntr}

\DeclareMathOperator*{\Res}{Res}

\renewcommand\thesection{\arabic{section}}
\renewcommand\thesubsection{\thesection.\arabic{subsection}}
\renewcommand\thesubsubsection{\thesubsection.\arabic{subsubsection}}

\renewcommand\thesectiondis{\arabic{section}}
\renewcommand\thesubsectiondis{\thesectiondis.\arabic{subsection}}
\renewcommand\thesubsubsectiondis{\thesubsectiondis.\arabic{subsubsection}}


\hyphenation{op-tical net-works semi-conduc-tor}
\def\inputGnumericTable{}                                 %%

\lstset{
%language=C,
frame=single, 
breaklines=true,
columns=fullflexible
}
\begin{document}

\newcommand{\BEQA}{\begin{eqnarray}}
\newcommand{\EEQA}{\end{eqnarray}}
\newcommand{\define}{\stackrel{\triangle}{=}}
\bibliographystyle{IEEEtran}
\raggedbottom
\setlength{\parindent}{0pt}
\providecommand{\mbf}{\mathbf}
\providecommand{\pr}[1]{\ensuremath{\Pr\left(#1\right)}}
\providecommand{\qfunc}[1]{\ensuremath{Q\left(#1\right)}}
\providecommand{\sbrak}[1]{\ensuremath{{}\left[#1\right]}}
\providecommand{\lsbrak}[1]{\ensuremath{{}\left[#1\right.}}
\providecommand{\rsbrak}[1]{\ensuremath{{}\left.#1\right]}}
\providecommand{\brak}[1]{\ensuremath{\left(#1\right)}}
\providecommand{\lbrak}[1]{\ensuremath{\left(#1\right.}}
\providecommand{\rbrak}[1]{\ensuremath{\left.#1\right)}}
\providecommand{\cbrak}[1]{\ensuremath{\left\{#1\right\}}}
\providecommand{\lcbrak}[1]{\ensuremath{\left\{#1\right.}}
\providecommand{\rcbrak}[1]{\ensuremath{\left.#1\right\}}}
\theoremstyle{remark}
\newtheorem{rem}{Remark}
\newcommand{\sgn}{\mathop{\mathrm{sgn}}}
\newcommand{\comb}[2]{{}^{#1}\mathrm{C}_{#2}}
\providecommand{\abs}[1]{\vert#1\vert}
\providecommand{\res}[1]{\Res\displaylimits_{#1}} 
\providecommand{\norm}[1]{\lVert#1\rVert}
%\providecommand{\norm}[1]{\lVert#1\rVert}
\providecommand{\mtx}[1]{\mathbf{#1}}
\providecommand{\mean}[1]{E[ #1 ]}
\providecommand{\fourier}{\overset{\mathcal{F}}{ \rightleftharpoons}}
%\providecommand{\hilbert}{\overset{\mathcal{H}}{ \rightleftharpoons}}
\providecommand{\system}{\overset{\mathcal{H}}{ \longleftrightarrow}}
	%\newcommand{\solution}[2]{\textbf{Solution:}{#1}}
\newcommand{\solution}{\noindent \textbf{Solution: }}
\newcommand{\cosec}{\,\text{cosec}\,}
\providecommand{\dec}[2]{\ensuremath{\overset{#1}{\underset{#2}{\gtrless}}}}
\newcommand{\solution}{\noindent \textbf{Solution: }}
\newcommand{\myvec}[1]{\ensuremath{\begin{pmatrix}#1\end{pmatrix}}}
\newcommand{\mydet}[1]{\ensuremath{\begin{vmatrix}#1\end{vmatrix}}}
\numberwithin{equation}{subsection}
\makeatletter
\@addtoreset{figure}{problem}
\makeatother
\let\StandardTheFigure\thefigure
\let\vec\mathbf
\renewcommand{\thefigure}{\theproblem}
\def\putbox#1#2#3{\makebox[0in][l]{\makebox[#1][l]{}\raisebox{\baselineskip}[0in][0in]{\raisebox{#2}[0in][0in]{#3}}}}
     \def\rightbox#1{\makebox[0in][r]{#1}}
     \def\centbox#1{\makebox[0in]{#1}}
     \def\topbox#1{\raisebox{-\baselineskip}[0in][0in]{#1}}
     \def\midbox#1{\raisebox{-0.5\baselineskip}[0in][0in]{#1}}
\vspace{3cm}
\title{Assignment 7}
\author{Dontha Aarthi-CS20BTECH11015}
\maketitle
\newpage
\bigskip
\renewcommand{\thefigure}{\theenumi}
\renewcommand{\thetable}{\theenumi}


Download latex-tikz codes from 

\begin{lstlisting}
https://github.com/Dontha-Aarthi/AI1103/blob/main/Assignment7/assignment7.tex
\end{lstlisting}
\begin{center}
  \section{\textbf{CSIR UGC NET EXAM (Dec 2014), Q.108}} 
\end{center}
    $N,A_1,A_2\cdots$ are independent real valued random variables such that 
    \begin{align}
        \pr{N=k}=(1-p)p^k,k=0,1,2,3\cdots
    \end{align}
    where $0<p<1$ and $\{A_i:i=1,2,\cdots\}$ is a sequence of independent and identically distributed bounded random variables. Let 
    \begin{align}
        X(w) = 
        \begin{cases}
        0  & \text{if } N(w)=0\\
        \sum_{j=1}^{k} A_j & \text{if } N(w)=k,k=1,2,3\cdots 
        \end{cases}
    \end{align}
    Which of the following are necessarily correct?\\
    \begin{enumerate}
        \item $X$ is a bounded random variable. 
        \item Moment generating function $m_X$ of $X$ is
        \begin{align}
            m_X(t)=\dfrac{1-p}{1-pm_A(t)}, t\in \mathbb{R},
        \end{align}
        where $m_A$ is moment generating function of $A_1$.
        \item Characteristic function $\varphi_X$ of $X$ is
        \begin{align}
            \varphi_X(t)=\dfrac{1-p}{1-p\varphi_A(t)},t\in \mathbb{R},
        \end{align}
        where $\varphi_A$ is the characteristic function of $A_1$.
        \item $X$ is symmetric about 0.
    \end{enumerate}
\section{Solution}
\begin{enumerate}
    \item Since $k$ is not bounded, $X$ cannot be a bounded random variable necessarily.\\
$\therefore$ \textbf{option 1 is incorrect}.\\
\item The moment generating function of a random variable is defined as
\begin{align}
    M_X(t)=E[e^{tX}]
\end{align}
So,
\begin{align}
    M_X(t)&=\sum_{k=0}^{\infty}\left(E[e^{t\sum_{j=1}^{k} A_j}
    |N=k]\times \pr{N=k}\right)\\
    &\begin{multlined}
    =E[e^{t\times 0}]\times \pr{N=0}+\\
    \sum_{k=1}^{\infty}\left(E[e^{t\sum_{j=1}^{k} A_j}
    |N=k]\times \pr{N=k}\right)
    \end{multlined}\\
    &\begin{multlined}
    =E[1]\times (1-p)+\\
    \sum_{k=1}^{\infty}\left(E[e^{t\sum_{j=1}^{k} A_j}|N=k]\times \pr{N=k}\right) \label{eq_3}
    \end{multlined}
\end{align}
Since $A_1,A_2\cdots$ are identical and independent, \eqref{eq_3} can be written as
\begin{align}
    &=(1-p)+\sum_{k=1}^{\infty}\left(\pr{N=k}\times\prod_{j=1}^k E[e^{tA_j}]\right) \\
    &=(1-p)+\sum_{k=1}^{\infty}\left(\pr{N=k}\times\prod_{j=1}^k M_{A_j}(t)\right) \\
   & =(1-p)+\sum_{k=1}^{\infty}\left((1-p)p^k\times(M_{A}(t))^k\right)\;\label{eq_4}
\end{align}
where, $M_{A}(t)$ is moment generating function of $A_1$.\\
On simplifying sum of infinite terms in geometric progression in \eqref{eq_4}, we get
\begin{align}
    &=(1-p)\times \left(1+\frac{p\times M_A(t)}{1-p\times M_A(t)}\right)\\
    &=\frac{1-p}{1-p\times M_A(t) }
\end{align}
$\therefore$ \textbf{option 2 is correct}.\\

\item The characteristic function of a random variable is defined as
\begin{align}
    \varphi_X(t)=E[e^{itX}]
\end{align}
And using this,
\begin{align}
    E[X]=\sum_{y} E[X|Y=y]\times \pr{Y=y}
\end{align}
We get 
\begin{align}
    \varphi_X(t)&=\sum_{k=0}^{\infty}\left(E[e^{it\sum_{j=1}^{k} A_j}
    |N=k]\times \pr{N=k}\right)\\
    &\begin{multlined}
    =E[e^{it\times 0}]\times \pr{N=0}+\\
    \sum_{k=1}^{\infty}\left(E[e^{it\sum_{j=1}^{k} A_j}
    |N=k]\times \pr{N=k}\right)
    \end{multlined}\\
    &\begin{multlined}
    =E[1]\times (1-p)+\\
    \sum_{k=1}^{\infty}\left(E[e^{it\sum_{j=1}^{k} A_j}|N=k]\times \pr{N=k}\right) \label{eq_0}
    \end{multlined}\\
\end{align}
Since $A_1,A_2\cdots$ are independent, \eqref{eq_0} can be written as
\begin{align}
    &=(1-p)+\sum_{k=1}^{\infty}\left(\pr{N=k}\times\prod_{j=1}^k E[e^{itA_j}]\right) \\
    &=(1-p)+\sum_{k=1}^{\infty}\left(\pr{N=k}\times\prod_{j=1}^k \varphi_{A_j}(t)]\right) \label{eq_1}
\end{align}
Since $A_1,A_2\cdots$ are identical, \eqref{eq_1} can be written as
\begin{align}
    =(1-p)+\sum_{k=1}^{\infty}\left((1-p)p^k\times(\varphi_{A}(t))^k\right)\;\label{eq_2}
\end{align}
where, $\varphi_{A}(t)$ is characteristic function of $A_1$.\\
On simplifying sum of infinite terms in geometric progression in \eqref{eq_2}, we get
\begin{align}
    &=(1-p)\times \left(1+\frac{p\times \varphi_A(t)}{1-p\times \varphi_A(t)}\right)\\
    &=\frac{1-p}{1-p\times \varphi_A(t) }
\end{align}
$\therefore$ \textbf{option 3 is correct}.\\

\item Now, lets find mean of $X$.
\begin{align}
    \text{Mean(X)}&=\frac{\sum_{j=1}^{k} A_j}{k}
\end{align}
For a distribution to be symmetric about a, the mean and the median should be equal to a, and with the given conditions it is not necessary for the mean($X$) and median of $X$ to be equal to 0, therefore $X$ is not necessarily symmetric about 0.\\
$\therefore$ \textbf{option 4 is incorrect}.
\end{enumerate}
$\therefore$ The correct options are (2) and (3).
\end{document}




