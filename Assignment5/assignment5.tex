\documentclass[journal,12pt,twocolumn]{IEEEtran}

\usepackage{setspace}
\usepackage{gensymb}
\singlespacing
\usepackage[cmex10]{amsmath}

\usepackage{amsthm}

\usepackage{mathrsfs}
\usepackage{txfonts}
\usepackage{stfloats}
\usepackage{bm}
\usepackage{cite}
\usepackage{cases}
\usepackage{subfig}

\usepackage{longtable}
\usepackage{multirow}

\usepackage{enumitem}
\usepackage{mathtools}
\usepackage{steinmetz}
\usepackage{tikz}
\usepackage{circuitikz}
\usepackage{verbatim}
\usepackage{tfrupee}
\usepackage[breaklinks=true]{hyperref}
\usepackage{graphicx}
\usepackage{tkz-euclide}

\usetikzlibrary{calc,math}
\usepackage{listings}
    \usepackage{color}                                            %%
    \usepackage{array}                                            %%
    \usepackage{longtable}                                        %%
    \usepackage{calc}                                             %%
    \usepackage{multirow}                                         %%
    \usepackage{hhline}                                           %%
    \usepackage{ifthen}                                           %%
    \usepackage{lscape}     
\usepackage{multicol}
\usepackage{chngcntr}

\DeclareMathOperator*{\Res}{Res}

\renewcommand\thesection{\arabic{section}}
\renewcommand\thesubsection{\thesection.\arabic{subsection}}
\renewcommand\thesubsubsection{\thesubsection.\arabic{subsubsection}}

\renewcommand\thesectiondis{\arabic{section}}
\renewcommand\thesubsectiondis{\thesectiondis.\arabic{subsection}}
\renewcommand\thesubsubsectiondis{\thesubsectiondis.\arabic{subsubsection}}


\hyphenation{op-tical net-works semi-conduc-tor}
\def\inputGnumericTable{}                                 %%

\lstset{
%language=C,
frame=single, 
breaklines=true,
columns=fullflexible
}
\begin{document}

\newcommand{\BEQA}{\begin{eqnarray}}
\newcommand{\EEQA}{\end{eqnarray}}
\newcommand{\define}{\stackrel{\triangle}{=}}
\bibliographystyle{IEEEtran}
\raggedbottom
\setlength{\parindent}{0pt}
\providecommand{\mbf}{\mathbf}
\providecommand{\pr}[1]{\ensuremath{\Pr\left(#1\right)}}
\providecommand{\qfunc}[1]{\ensuremath{Q\left(#1\right)}}
\providecommand{\sbrak}[1]{\ensuremath{{}\left[#1\right]}}
\providecommand{\lsbrak}[1]{\ensuremath{{}\left[#1\right.}}
\providecommand{\rsbrak}[1]{\ensuremath{{}\left.#1\right]}}
\providecommand{\brak}[1]{\ensuremath{\left(#1\right)}}
\providecommand{\lbrak}[1]{\ensuremath{\left(#1\right.}}
\providecommand{\rbrak}[1]{\ensuremath{\left.#1\right)}}
\providecommand{\cbrak}[1]{\ensuremath{\left\{#1\right\}}}
\providecommand{\lcbrak}[1]{\ensuremath{\left\{#1\right.}}
\providecommand{\rcbrak}[1]{\ensuremath{\left.#1\right\}}}
\theoremstyle{remark}
\newtheorem{rem}{Remark}
\newcommand{\sgn}{\mathop{\mathrm{sgn}}}
\newcommand{\comb}[2]{{}^{#1}\mathrm{C}_{#2}}
\providecommand{\abs}[1]{\vert#1\vert}
\providecommand{\res}[1]{\Res\displaylimits_{#1}} 
\providecommand{\norm}[1]{\lVert#1\rVert}
%\providecommand{\norm}[1]{\lVert#1\rVert}
\providecommand{\mtx}[1]{\mathbf{#1}}
\providecommand{\mean}[1]{E[ #1 ]}
\providecommand{\fourier}{\overset{\mathcal{F}}{ \rightleftharpoons}}
%\providecommand{\hilbert}{\overset{\mathcal{H}}{ \rightleftharpoons}}
\providecommand{\system}{\overset{\mathcal{H}}{ \longleftrightarrow}}
	%\newcommand{\solution}[2]{\textbf{Solution:}{#1}}
\newcommand{\solution}{\noindent \textbf{Solution: }}
\newcommand{\cosec}{\,\text{cosec}\,}
\providecommand{\dec}[2]{\ensuremath{\overset{#1}{\underset{#2}{\gtrless}}}}
\newcommand{\solution}{\noindent \textbf{Solution: }}
\newcommand{\myvec}[1]{\ensuremath{\begin{pmatrix}#1\end{pmatrix}}}
\newcommand{\mydet}[1]{\ensuremath{\begin{vmatrix}#1\end{vmatrix}}}
\numberwithin{equation}{subsection}
\makeatletter
\@addtoreset{figure}{problem}
\makeatother
\let\StandardTheFigure\thefigure
\let\vec\mathbf
\renewcommand{\thefigure}{\theproblem}
\def\putbox#1#2#3{\makebox[0in][l]{\makebox[#1][l]{}\raisebox{\baselineskip}[0in][0in]{\raisebox{#2}[0in][0in]{#3}}}}
     \def\rightbox#1{\makebox[0in][r]{#1}}
     \def\centbox#1{\makebox[0in]{#1}}
     \def\topbox#1{\raisebox{-\baselineskip}[0in][0in]{#1}}
     \def\midbox#1{\raisebox{-0.5\baselineskip}[0in][0in]{#1}}
\vspace{3cm}
\title{Assignment 5}
\author{Dontha Aarthi-CS20BTECH11015}
\maketitle
\newpage
\bigskip
\renewcommand{\thefigure}{\theenumi}
\renewcommand{\thetable}{\theenumi}


Download latex-tikz codes from 

\begin{lstlisting}
https://github.com/Dontha-Aarthi/AI1103/blob/main/Assignment5/assignment5.tex
\end{lstlisting}
\begin{center}
  \section{\textbf{GATE 2021 (ST), Q.48 (statistics section)}} 
\end{center}
    Let $\{X_n\}_{n\ge 1}$ be a sequence of independent and identically distributed random variables each having uniform distribution on [0,3]. Let $Y$ be a random variable, independent of $\{X_n\}_{n\ge 1}$, having probability mass function
\begin{align}
\pr{Y=k} = 
\begin{cases}
\frac{1}{(e-1)k!} & k=1,2,3\cdots \\
0 & otherwise
\end{cases}
\end{align}
Then $\pr{max\{X_1,X_2,\cdots X_Y\}\le 1}$ equals ............\\
\section{Solution}
Given that $\{X_n\}_{n\ge 1}$ is having a uniform distribution on [0,3], so probability can be written as
\begin{align}
\pr{X_n}_{n\ge 1}=  
\begin{cases}
\frac{1}{3} & 0\le X_n\le 3 \\
0 & otherwise
\end{cases}
\end{align}
So,
\begin{align}
    \pr{X_n\le 1}_{n\ge 1}=\frac{1}{3}
\end{align}
Required probability
\begin{align}
 &=\pr{max\{X_1,X_2,\cdots X_Y\}\le 1}
 \end{align}
 Since, $\{X_n\}_{n\ge 1}$ is a sequence of independent variables and $Y$ is also independent of $\{X_n\}_{n\ge 1}$ so, required probability
 \begin{align}
& \begin{multlined}
    =\pr{max\{X_1\}\le 1}\cdot \pr{Y=1}+\\
    \pr{max\{X_1,X_2\}\le 1}\cdot \pr{Y=2}+\cdots \infty 
\end{multlined}\\
&\begin{multlined}
    =\pr{X_1\le 1}\cdot \pr{Y=1}+\\
    \pr{X_1,X_2\le 1}\cdot\pr{Y=2}+\cdots \infty
\end{multlined}\\
&\begin{multlined}
    =\pr{X_1\le 1}\cdot \pr{Y=1}+\\
    \pr{X_1\le 1}\cdot\pr{X_2\le 1}\cdot\pr{Y=2}+\cdots \infty
\end{multlined}
\end{align}
\begin{align}
&\begin{multlined}
=\frac{1}{3}\cdot\frac{1}{(e-1)\cdot 1!}+\\
\frac{1}{3}\cdot \frac{1}{3}\cdot\frac{1}{(e-1)\cdot 2!}+\cdots \infty
\end{multlined}\\
&=\sum_{p=1}^{\infty}\left(\frac{1}{e-1}\right)\left(\frac{1}{3}\right)^p\left(\frac{1}{p!}\right)\\
&=\left(\frac{1}{e-1}\right)\left[\sum_{p=0}^{\infty}\left(\frac{1}{3}\right)^p\left(\frac{1}{p!}\right)-1\right]\label{taylor}
\end{align}
Using Taylor's Series of $e^x$ in \eqref{taylor}, \\
Required probability
\begin{align}
    &=\frac{e^{1/3}}{e-1}-\frac{1}{e-1}\\
    &=0.23
\end{align}
    



\end{document}




